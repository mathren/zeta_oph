\documentclass[twocolumn,twocolappendix,trackchanges]{aastex63}
\usepackage{amsmath}
\newcommand{\code}[1]{\texttt{#1}}
\newcommand{\mesa}{\code{MESA}}
\newcommand{\MESA}{\code{MESA}}
\renewcommand{\labelitemii}{$\bullet$}
\newcommand{\kms}{{\mathrm{km\ s^{-1}}}}
\newcommand{\Msun}{{\mathrm{M}_\odot}}
\newcommand{\kev}{\mathrm{keV}}
\newcommand{\gk}{\ensuremath{\,\rm{GK}}}
\usepackage{CJK}
\DeclareRobustCommand{\Eqref}[1]{Eq.~\ref{#1}}
\DeclareRobustCommand{\Figref}[1]{Fig.~\ref{#1}}
\DeclareRobustCommand{\Tabref}[1]{Tab.~\ref{#1}}
\DeclareRobustCommand{\Secref}[1]{Sec.~\ref{#1}}
\newcommand{\zoph}{$\zeta$ Oph}

\newcommand{\todo}[1]{{\large $\blacksquare$~\textbf{\color{red}[#1]}}~$\blacksquare$}

\begin{document}

\graphicspath{{./figures/}}


\title{Testing models of accreting stars in
  massive binaries on $\zeta$ Ophiuchi}
\author[0000-0002-6718-9472]{M.~Renzo}
\affiliation{Department of Physics, Columbia University, New York, NY 10027, USA}
\affiliation{Center for Computational Astrophysics, Flatiron Institute, New York, NY 10010, USA}

\author{Y.~G\"otberg}
\affiliation{The Observatories of the Carnegie Institution for Science, 813 Santa Barbara Street, Pasadena, CA 91101, USA}

\author{\todo{TBD}}

\begin{abstract}
  Binarity dominates the evolution of massive stars, and the nearest
  O-type star to Earth, $\zeta$ Ophiuchi, has long been proposed to be
  a product of binary evolution. Despite this, most stellar models
  have tried unsuccessfully to reproduce its observable properties
  relying on single-star rotating models.  \todo{Here we do better}
\end{abstract}

\vspace*{-10pt}
\keywords{stars: individual: $\zeta$ Ophiuchi  -- stars: massive --
  stars: binaries} %% check keywords exist


%% main text limit:
%% 3500 words, 50 references, 5 figures (each up to 9 panels)
\section{Introduction}
\label{sec:intro}

The overwhelming majority of massive stars is born in multiple systems
\citep[e.g.,][]{mason:09, almeida:17}, and a large
fraction will exchange mass or merge with a companion in their lifetime \citep[e.g.,][]{sana:12}. The most common type of interaction is a post-main-sequence
stable mass transfer (case B) through Roche Lobe overflow
(RLOF, \citealt{kippenhahn:67}, \todo{pop synth. ref.?}.
Many studies \citep[e.g.][]{gotberg:17, gotberg:18,
  laplace:20, laplace:21} have focused on the dramatic
impact of these interactions on the donor star, often treating the
closest binary companion as a point mass. \todo{more refs, other groups}

However, binary interactions have a crucial impact on the secondary
star too. Because of mass transfer, these are expected to accrete mass
and rejuvenate because of the accompanying growth of the convective
core \citep[e.g.,][]{neo:77, schneider:16}, spin up to critical
rotation \citep[e.g.,][]{packet:81, cantiello:07}, and
possibly be polluted by CNO-processed material from the inner core of
the donor star \citep[e.g.,][]{blaauw:93}.

Understanding the evolution of accretors in massive binaries has wider
and crucial implications for stellar populations, electromagnetic
transient observations, and gravitational-wave progenitors. Binary
products, and accretors in particular, can impact cluster populations
and their age estimates and main sequence morphology
\citep[e.g.,][]{pols_marinus:94, wang:20}. Moreover, the majority of
massive binaries will be disrupted by the first supernova ejecting the
companion \citep[``binary SN scenario'', ][]{blaauw:61, dedonder:97,
  eldridge:11, renzo:19walk, evans:20}. Therefore, populations of
field massive stars contain presently single O-type stars that
accreted mass earlier on. The majority of these will be too slow to
stand out in astrometric surveys \citep[e.g.,][]{eldridge:11,
  renzo:19walk}. Assuming a constant star formation
history, \cite{renzo:19walk} estimated that $10.1^{+4.6}_{-8.6}\%$ of
O-type stars might be accretors released after a SN -- where the
errors span a range of parameter variations.

From the transients
perspective, accretors stars are also important: \cite{zapartas:19}
showed that $14_{-11}^{+4}\%$ of hydrogen-rich (type II) SNe might
come from these progenitors after being ejected from a binary. The
fact that they accreted mass before exploding can influence their
helium (He) core mass and thus the explosion properties and the
inferred progenitors \citep{zapartas:21}.

Finally, the majority of
isolated binary evolutionary scenarios for gravitational-wave
progenitors go through a common-envelope phase initiated by the
originally less massive accretor after the formation of the first
compact object \citep[e.g.,][]{belczynski:16nat, tauris:17,
  broekgaarden:21}. Therefore, it is possible that accretion of mass
before the formation of the first compact object could modify the
internal structure of the star that will initiate the common-envelope
phase \citep[e.g.][]{law-smith:20, klencki:21}.

Despite their importance, accretor stars in binaries have so far
received much less attention than the donor stars, with the pioneering
work of \cite{hellings:83, hellings:84} and \cite{braun:95} as notable
exceptions. Large grids of accretor models are missing, and only
sparse models exist \citep[e.g.,][]{cantiello:07}\todo{more
  refs.}. This is because of the complexity of these models, where one
needs to follow in detail the \emph{coupled} evolution of two rotating
stars exchanging mass. Moreover, the admittedly large number of free
parameters involved in the modeling of each individual star and their
interactions makes robust predictions challenging to obtain. Here, we
will argue that the nearest O-type star to Earth, $\zeta$
Ophiuchi\footnote{also known as HD\,149\,757.} (\zoph) provides a unique
opportunity to constrain these models.

\zoph\ has a distance from Earth of $107\pm4$\,pc \citep[][and
references therein]{neuhauser:20}, and a spectral type O9.5{\rm IVnn}
\citep{sota:14}. It occasionally shows emission lines
\citep{walker:79, vink:09}, making it an Oe star. It was originally
identified as a runaway because of its large proper motion by
\cite{blaauw:52}. Unfortunately, the \emph{Gaia} data for this object
are not of sufficient quality\footnote{The renormalized unit weighted
  error (RUWE) of this star in Gaia EDR3 is 4.48.} to improve previous astrometric results,
but estimates of the peculiar velocity range in $30-50\,\kms$
\citep[e.g.,][]{zehe:18, neuhauser:20}. The large velocity with
respect the surrounding interstellar material is also confirmed by the
presence of a prominent bow-shock \citep[e.g.,][]{bodensteiner:18}.

Because of its young apparent age, extremely fast rotation
($v\sin(i)\gtrsim 400\,\kms$, e.g., \citealt{zehe:18}), and nitrogen
(N) and He rich surface \citep[e.g.,][]{herrero:92, blaauw:93,
  villamariz:05, marcolino:09}, \zoph\ is a prime candidate for the
binary SN scenario \citep{blaauw:93}. Many studies have
suggested \zoph\ might have accreted mass from a companion before
acquiring its large velocity, both from spectroscopic and kinematic
considerations \citep[e.g.,][]{blaauw:93, hoogerwerf:00,
  hoogerwerf:01, tetzlaff:10, neuhauser:20} and using stellar modeling
arguments \citep[e.g.,][]{vanrensbergen:96}. Recently,
\cite{neuhauser:20} suggested that a supernova in
Upper-Centaurus-Lupus produced the pulsar PSR B1706-16, ejected \zoph,
and also injected the short-lived radioactive isotope
$^{60}\mathrm{Fe}$ on Earth $1.78\pm0.21$\,Myr ago. This argues
strongly for a successful supernova explosion accompanied by a large
$\sim 250\,\kms$ natal kick, which in most cases would be sufficient
to disrupt the binary.

Although the nature of \zoph\ as a binary product is well established,
its observed large surface rotation rate has lead previous attempts to
rely on rotational mixing to explain the surface composition
\cite[e.g.,][]{maeder:00}. Even the binary models of
\cite{vanrensbergen:96} assumed spin-up due to mass accretion to drive
rotational mixing from the interior of the accreting star (see also
\citealt{cantiello:07}). However, \cite{villamariz:05} were unable to
find good fit for the stellar spectra using the rotating models from
\cite{meynet:00}.

This may not be surprising: rotational mixing has lower efficiency for
metal-rich and relatively low mass stars because of the increased
importance of mean molecular weight gradients and longer thermal
timescales compared to more massive stars \citep[e.g.,][]{yoon:06,
  perna:14}. The parent association has a metallicity $Z=0.01\simeq Z_\odot$
\citep[based on asteroseismology from][]{murphy:21}, and mass estimates for \zoph\ range from
$13-25\,M_\odot$, at the lower end of the range where efficient mixing
might bring He and CNO-processed material to the surface (chemically
homogeneous evolution).

\todo{maybe paragraph below goes in discussion}
On top of the surface abundances, its extreme rotation rate, and the
peculiar space velocity, \zoph\ poses a number of other
puzzles: its wind mass-loss rate is about two orders of magnitude
lower than theoretical predictions (weak wind problem,
\citealt{marcolino:09}), the star exhibits spectral variability with
occasional appearance of H$\alpha$ in emission
\citep[e.g.,][]{walker:79}, and is potentially magnetic \todo{true?ref?}.

Given the challenges in explaining the surface composition of \zoph\
with rotational mixing from the stellar interior and the strong
evidence for its past as a member of a binary system, this star offers
a unique opportunity to constrain the evolution of accreting stars in
massive binary systems.

Here, we present self-consistent binary evolution models for
$\zeta$ Oph computing simultaneously the coupled evolution of
\emph{both} donor and accretor star and their orbit. \todo{fix order description} After presenting
our calculations in \Secref{sec:methods}, we show our best model which
reproduces the majority of the salient features of this star in
\Secref{sec:best_model}. In this model, the surface abundances of
\zoph\ are explained by pollution from the former companion, rather
than upward mixing from the interior of \zoph\ itself. We discuss the
sensitivity of our results to the admittedly many free parameters
required for this kind of computations in
\Secref{sec:param_variations}. Finally, we conclude in
\Secref{sec:conclusions}.



\section{ \texttt{MESA} modeling of massive binaries}
\label{sec:methods}

Modeling the evolution of massive binaries
($M_1\gtrsim 20\,M_\odot \geq M_2$) is challenging because of the
intricate role of several notoriously difficult stellar physics
ingredients (differential rotation, mixing, high mass-loss rates,
accretion, etc.). Here we follow self-consistently the coupled
evolution of two massive stars in a binary system using \texttt{MESA}
(version 15140). Our choice of input parameters and our numerical
results are available at \todo{link}. We discuss here only the main
relevant physical parameters, and Appendix~\ref{sec:software} gives
more details on our choice of input physics.

We adopt the Ledoux criterion to determine convective stability and a
mixing length parameter of $1.5$. We include semiconvection and
thermohaline mixing following \cite{langer:83} and
\cite{kippenhahn:80}, respectively, each with efficiency $1.0$. We use the exponential core overshooting from \cite{herwig:00}
with free parameters $(f, f_0)=(4.25\times10^{-2}, 10^{-3})$
\citep{claret:17} which broadly reproduce the width of the main
sequence from \cite{brott:11}. We also use the local implicit
enhancement of the convective flux in superadiabatic regions
(\texttt{MLT--}) introduced in \texttt{MESA}
15140.

We treat rotation in the ``shellular'' approximation and initialize it
assuming tidal synchronization at the
beginning of the evolution. For our fiducial period choice ($P=100$\,days), this effectively means both the stars in our binary are
initially slow rotators. Our models include in a diffusive
approximation the effect of Eddington-Sweet circulations
\citep{sweet:50}, which dominates the chemical mixing due to
rotation. We also include the secular and dynamical shear
instabilities, and the Goldreich-Schubert-Fricke instability as in
\cite{gotberg:17, gotberg:18, laplace:20, laplace:21}.  We assume a Spruit-Tayler
dynamo for the transport of angular momentum \citep{spruit:02}, and chose the same free
parameters as \cite{heger:00}. This also includes the rotational
enhancement of wind mass loss as in \cite{langer:98}.

We treat wind mass loss with the \cite{vink:00,vink:01} hot wind,
and \cite{dejager:88} with a scaling factor of 1. This effectively
means our wind mass loss rate post-mass transfer is overestimated by
almost a factor of 100 \citep[weak wind problem, see][]{marcolino:09}.


Both stars are evolved simultaneously on the same timesteps until
after the donor detaches from the Roche lobe. We follow \cite{kolb:90}
to calculate the mass transfer rate from optically thick layers of the
donor star during Roche lobe overflow (RLOF). Moreover, we assume that
the transfered layers reach the accretor with the accretor's specific
surface angular momentum and entropy, but the chemical composition is
determined by the inner evolution of the donor star. Mass transfer is
conservative until the accretor reaches critical rotation, after which
rotationally enhanced mass loss governs the accretion efficiency.

To define RLOF detachment, we take advantage of the fact that we focus
here on case B interactions among massive stars. After losing its
envelope, massive donors are not expected to expand to hundreds of
$R_\odot$ during He shell burning at the metallicity we consider
\citep[e.g.,][]{laplace:20}. Thus, we define RLOF detachment as the moment
after the onset of RLOF when the donor has a surface He mass fraction
larger than 0.35 (indicating that a significant amount of envelope has
been lost or transferred), a radius smaller than its terminal-age main
sequence (TAMS) radius, and no mass is being transferred anymore. At
this point in time, we save a model for the accreting star, and
continue its evolution as a single star until TAMS with the same
setup.

\todo{describe parameter variations in this sec.}

\section{Massive binary evolution naturally explains $\zeta$
  Ophiuchi's properties}
\label{sec:best_model}

\begin{figure}[bp]
  \includegraphics[width=0.5\textwidth]{HRD_both}
  \caption{HRD for the donor star (top) and accretor star (bottom) of
    the progenitor binary of \zoph. Each point is separated by 50
    years of evolution. The colors represent the
    stellar age, the red datapoint shows the position of \zoph\
    according to \cite{villamariz:05}, and the black diamonds mark the
    position at the end of the binary run. We continue the accretor
    evolution as a single star from there until core H depletion,
    hence the bottom panel shows a longer time. Note the different
    scales on the two panels. The thin gray dashed line show the main
    sequence evolution of non-rotating single stars of 15, 17, 25, and
    30\,$M_\odot$ at $Z=0.01$ for comparison.}
  \label{fig:HRD_both}
\end{figure}


We describe here the evolution of a binary system where the accretor
star can broadly reproduce all the observed features of \zoph. We assume initial masses
$M_1=25\,M_\odot$, $M_2=17\,M_\odot$ on a period of $100$\,days at
$Z=0.01$.

\Figref{fig:HRD_both} shows the Hertzsprung-Russell diagrams (HRD)
of both stars. After $7.24$\,Myr, the donor star (top panel) evolves off the main
sequence and $\sim8400$\,years later, when the donor's effective
temperature reaches about $T_\mathrm{eff}\simeq 10^4$\,K, mass transfer starts. This
results in a stable case B RLOF. We refer to \cite{gotberg:17, laplace:21, blagorodnova:21}
and references therein for a detailed description of the evolution of
massive donor stars in binaries. Although our models here are
more massive, the qualitative behavior of the donor star is similar.

At the onset of RLOF, the accretor star (bottom panel of
\Figref{fig:HRD_both}) is still on the main sequence with
$T_\mathrm{eff}\simeq10^{4.5}$\,K. Because of accretion,
it quickly becomes over-luminous ($L\simeq10^{5.4}\,L_\odot$), and its
radius increases dramatically from $\sim7.5\,R_\odot$ to
$\sim35\,R_\odot$. Once the accretor reaches critical rotation
(roughly at the lowest $T_\mathrm{eff}$ in the bottom panel of
\Figref{fig:HRD_both}), the star begins contracting and its
$T_\mathrm{eff}$ increases. At $T_\mathrm{eff}\simeq 4.{43}$\,K the
material transferred from the companion star becomes progressively
more He-rich, causing a ``v-shaped'' feature in the evolutionary
track. This indicates that the outer layers of the donor core are
uncovered by mass transfer, after the convective core recession in
mass during the main sequence. On top of modifying the morphology of
the evolutionary track, this late mass transfer puts material at high
mean molecular weight $\mu$ on top of the primordial envelope of the
accretor. This also starts vigorous thermohaline mixing in the
accreting star, which, together with rotational mixing, progressively
dilutes the surface He mass fraction and causes noisy features on the
HR diagram \citep[e.g.,][]{cantiello:07}.  \Secref{sec:mixing}
describes in more detail the mixing processes inside the accretor, we
emphasize here that the algorithmic choices made to model mixing might
impact the morphology of the accretor's evolutionary track during
RLOF.

We evolve the binary system until the black diamonds in
\Figref{fig:HRD_both}, which occurs well after the donor detaches from
the Roche Lobe. At this point, the accretor is a H-rich fast-rotating
star of
$\sim$$20.1\,M_\odot$. Available mass estimates for the presently single \zoph\ are highly uncertain, but most include
$20\,M_\odot$ \citep[e.g.,][]{hoogerwerf:01, villamariz:05, neuhauser:20}. Its post-RLOF orbital velocity is
$v_2\simeq40\,\kms$, which is expected to decrease a bit further due to wind-driven widening of the binary, but is in good agreement with the presently observed runaway velocity of \zoph.

Accounting for both wind mass loss and the amount of mass transferred, at the end of RLOF the donor becomes a He star of
$\sim$$9.4\,M_\odot$, likely to contract further and appear as a
Wolf-Rayet star. It's surface H mass fraction is $\lesssim 0.2$ and
most of the H is likely to be removed by further wind mass loss
\citep[e.g.,][]{gotberg:17}. \todo{Ylva: want to expand?}.
For our assumed scenario to work, such donor needs to successfully
explode in a SN, breaking the binary system and making a neutron star
remnant. While the post-RLOF donor mass we obtain is rather high, recent studies
suggest higher ``explodability'' of donor stars in binary systems
\citep[e.g.,][]{schneider:20, laplace:21, vartanyan:21}. Furthermore, we
emphasize that neither the initial donor mass nor the initial period are
observable, and thus there is room to chose different values to obtain
easier to explode donors and faster (or slower) accretors post-RLOF.

From the black diamond onwards, we evolve the accretor as a single star with the same \texttt{MESA} setup until TAMS. The main-sequence track on which the accretor settles post-RLOF has a higher luminosity compared to the original track because of the accretion of mass, and it has also a slightly different morphology due to the accretion of matter partially processed in the core of the donor before mass transfer. The red point with errors in the bottom panel of \Figref{fig:HRD_both} marks the approximate position of \zoph based on the analysis of \cite{villamariz:05}. The color of the track in \Figref{fig:HRD_both} indicate that our accreting star spends about
$\sim$2\,Myr within the represented errorbars after the end of RLOF. Assuming the kinematic age of
$1.78\pm0.21$ \citep{neuhauser:20}, and estimating a remaining lifetime of the donor of
$\sim$0.5\,Myr, this gives the correct timescale for the binary SN scenario.

We note that the observed position of \zoph\ on the HRD, and especially
its relatively high $T_\mathrm{eff}$ would be hard to reproduce
assuming initially less massive accretors (which would remain too cool
even after accreting mass), or more equal initial mass ratio (which
would produce a too evolved accretor at the onset of mass transfer).

\begin{figure}[htbp]
  \includegraphics[width=0.5\textwidth]{MT}
  \caption{Mass transfer rates as a function of time during RLOF. The top (bottom) panel
    shows the donor (accretor) star. The cyan solid lines show the
    mass transfer rate between the two stars. The dashed blue lines
    show the actual change in the mass of the stars (due to the
    combination of wind, and accretion efficiency). The thin red
    lines show the wind mass loss rates. During RLOF the accretor
    reaches critical rotation, which leads to oscillations in the
    rotationally-enhahnced wind mass loss.}
  \label{fig:MT}
\end{figure}



\Figref{fig:MT} shows the rate of mass loss/accretion in each star
during RLOF. The top panel focuses on the donor star
which loses mass to RLOF (cyan line) and wind mass loss (thin red
line). The dashed blue lines show their combination resulting in the
actual rate of mass change of the stars.
The bottom panel shows instead the accreting star, which grows in mass
because of the mass transfer. The entire duration of this case B stable RLOF
is only about $10^4$\,years, and the mass transfer rates reaches very high
values above $10^{-2.5}\,M_\odot\ \mathrm{yr^{-1}}$.

During RLOF, the total amount of mass lost by the donor is
$\Delta M_\mathrm{donor} \simeq 10.6\,M_\odot$, of which only
$\Delta M_\mathrm{accretor}\simeq 3.4\,M_\odot$ are successfully
accreted. This corresponds to an overall mass
transfer efficiency
$\beta_\mathrm{RLOF}\equiv \Delta M_\mathrm{accretor}/\Delta M_\mathrm{donor} \simeq 0.3$,
although the accretion efficiency is \emph{not} constant throughout the
mass transfer, instead it depends on the radial and rotational evolution of
the accreting star.

At the end of RLOF, the donor star briefly expands again
($T_\mathrm{eff}\simeq10^{4.1}$\,K, $L\simeq10^{5.5}\,L_\odot$). This
is due to the partial recombination of the He rich material now at the
surface, which causes a transient surface convection layer. This
causes the second broad peak in the mass transfer rates seen at
$7.261$\,Myrs. We find this to be the culprit of difficulties in
modeling massive binaries transferring mass in older \texttt{MESA}
releases, because although only a very small amount of mass is
involved, this would lead to large radial expansion much beyond the
donor's Roche lobe, and cause numerical problems.

The wind mass loss (in red) controls the accretion efficiency and thus
the difference between the actual rate of change in mass of the
accretor (thick dashed blue) and the rate at which mass is being
transferred. At peak, where the red and the cyan lines overlap, mass
transfer becomes very non-conservative, but for most of the evolution
the (rotationally enhanced) wind removes only a fraction of the
accreted mass. The interplay between the stellar radius and rotation
causes the oscillations visible in the bottom panel, whose amplitude
is generally lower than the RLOF mass transfer rate.




\subsection{Internal mixing and angular momentum transport in the accretor}
\label{sec:mixing}

 \todo{clarify rotational mixing
  dominates, show mixing plot}

\begin{figure*}[htbp]
  \includegraphics[width=\textwidth]{D_mix}
  \caption{Internal mixing in the accreting star during the RLOF. The
    y axis shows the total diffusion coefficient (thick cyan line),
    and the contribution from processes unrelated to rotation (think
    blue line). These include convection (shown in red), overshooting
    (visible above the convective core), thermohaline mixing (pink)
    and semiconvection (in purple).}
  \label{fig:D_mix}
\end{figure*}



\todo{discuss rotation rate and radius with fig. 3}



\begin{figure}[htbp]
  \includegraphics[width=0.5\textwidth]{zeta_rot}
  \caption{Surface averaged rotation rate for the accretor
    model. Shortly after $\sim$7\,Myr the mass transfer quickly spins
    up the accretor at critical rotation. By the time the donor
    detaches from the RLOF the accretor is still spinning at
    $\sim$$400\,\kms$. At this point (beginning of the dot-dashed line), we continue the evolution as a single star, and the accretor quickly spins down. Note however that we use a wind mass-loss rate from \cite{vink:01}, which is observed to be
    $\sim$2 orders of magnitude too high.}
  \label{fig:rot}
\end{figure}



\section{Robustness of the model}
\label{sec:param_variations}

In this section we investigate the sensitivity of our results to
physical parameters.

\todo{How to present results? Table? Showing what? surface mass
  fractions, rotation, L, Teff}

\todo{
  Binary parameters:
  \begin{itemize}
  \item $M_1$
  \item $M_2$
  \item $P$
  \item J-accretion
  \end{itemize}
}

\todo{
  Single star parameters:
  \begin{itemize}
  \item thermohaline mixing
  \item Eddington-Sweet circulations
  \item metallicity
  \end{itemize}
}

\todo{others?}

\section{Discussion}
\label{sec:discussio}

\todo{no SN pollution/out of eq., but \cite{hirai:18} shows this is a
  small effect. Hirai+18 gives relaxation times}

\section{Conclusions}
\label{sec:conclusions}

We have demonstrated that self-consistent one-dimensional calculations
of coupled stellar models with masses $\gtrsim 20\,M_\odot$ are
possible with the \texttt{MESA} software instrument. As a first
application, we focused on finding a model for \zoph, assuming its
runaway nature is explained by the binary SN scenario.

We found that it is likely possible to explain its surface composition
without assuming that the surface excess of He and N comes from within
the star. Instead, this material comes from the receeding core of the
donor star. Therefore, the present day abundances constrain the
accretion efficiency and mixing in the accretor.

\zoph\ should therefore \emph{not} be used to test models of
rotational mixing in single star evolution, nor its more extreme
version of chemically homogeneous evolution.

\software{
  \texttt{mesaPlot} \citep{mesaplot},
  \texttt{mesaSDK} \citep{mesasdk},
  \texttt{ipython/jupyter} \citep{ipython},
  \texttt{matplotlib} \citep{matplotlib},
  \texttt{NumPy} \citep{numpy},
  \MESA \citep{paxton:11,paxton:13,paxton:15,paxton:18,paxton:19}
}

\acknowledgements{We are grateful to E.~Zapartas, A.~Jermyn,
  M.~Cantiello for helpful discussions.}



\appendix

\section{\texttt{MESA} setup}
\label{sec:software}

\todo{MLT--?}

\todo{possibly move to methods}
We use \code{MESA} version 15140 to compute our models.  The
\code{MESA} equation of state (EOS) is a blend of the OPAL \citet{Rogers2002}, SCVH
\citet{Saumon1995}, PTEH \citet{Pols1995}, HELM \citet{Timmes2000},
and PC \citet{Potekhin2010} EOSes. \todo{check if updated EOS?}

Radiative opacities are primarily from OPAL \citep{Iglesias1993,
  Iglesias1996}, with low-temperature data from \citet{Ferguson2005}
and the high-temperature, Compton-scattering dominated regime by
\citet{Buchler1976}. Electron conduction opacities are from
\citet{Cassisi2007}.

Nuclear reaction rates are a combination of rates from NACRE
\citep{Angulo1999}, JINA REACLIB \citep{Cyburt2010}, plus additional
tabulated weak reaction rates \citet{Fuller1985, Oda1994,
  Langanke2000}. Screening is included via the prescription of
\citet{Chugunov2007}.  Thermal neutrino loss rates are from
\citet{Itoh1996}. We use a
22-isotope nuclear network (\texttt{approx\_21\_plus\_cr56}).

We treat convection using the Ledoux criterion, and include
thermohaline mixing and semiconvection, both with an efficiency factor
of 1. We assume $\alpha_\mathrm{MLT}=1.5$ and use \todo{fix}
\cite{brott:11} overshooting for the convective core
burning. \todo{fix} Moreover, we employ the MLT++ artificial
enhancement of the convective flux \citep[e.g.,][]{paxton:15}. Stellar
winds are included using the algorithms from \cite{vink:01} with an
efficiency factor of
1.% and \cite{nugis:00} when the surface en abundance
% is below 0.4 and $T_\mathrm{eff}>10^{4.7}\,\mathrm{K}$. We use a
% hyperbolic tangent interpolation between the two for
% hydrogen-deficient surfaces with
% $10^{4.5}\,\mathrm{K}\leq T_\mathrm{eff}\leq10^{4.7}\,\mathrm{K}$. In
% both cases, we use an efficiency factor of 1, and our post-merger models never
% become cool enough to use a cool wind algorithm \citep[e.g.,][]{renzo:17}.

The inlists, processing scripts, and model output will be made available at~\href{link}{link}.


\section{Resolution tests}
\label{sec:res_tests}
\subsection{Spatial resolution}

\begin{figure*}[htbp]
  \centering
  \includegraphics[width=\textwidth]{spatial_res_plot}
  \caption{Left: HRD comparison for our fiducial binary model varying
  the number of mesh points. We only show the evolution until our definition
  of RLOF detachment. Right: number of mesh points as a
  function of timestep number. In both panels, the blue/cyan tracks show the donor stars, the
red/pink tracks show the accretor. Thicker dashed lines correspond to
the models at higher resolution (i.e., lower $\Delta$ which indicates
the value of \texttt{mesh\_delta\_coeff}).}
\end{figure*}


\bibliographystyle{aasjournal}
\bibliography{./zeta_ophiuchi.bib}

\end{document}

%%% Local Variables:
%%% mode: latex
%%% TeX-master: t
%%% End:
