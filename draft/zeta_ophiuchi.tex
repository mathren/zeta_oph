\documentclass[twocolumn,twocolappendix,trackchanges]{aastex63}
\usepackage{amsmath}
\newcommand{\code}[1]{\texttt{#1}}
\newcommand{\mesa}{\code{MESA}}
\newcommand{\MESA}{\code{MESA}}
\renewcommand{\labelitemii}{$\bullet$}
\newcommand{\kms}{{\mathrm{km\ s^{-1}}}}
\newcommand{\Msun}{{\mathrm{M}_\odot}}
\newcommand{\kev}{\mathrm{keV}}
\newcommand{\gk}{\ensuremath{\,\rm{GK}}}
\usepackage{CJK}
\DeclareRobustCommand{\Eqref}[1]{Eq.~\ref{#1}}
\DeclareRobustCommand{\Figref}[1]{Fig.~\ref{#1}}
\DeclareRobustCommand{\Tabref}[1]{Tab.~\ref{#1}}
\DeclareRobustCommand{\Secref}[1]{Sec.~\ref{#1}}

\newcommand{\todo}[1]{{\large $\blacksquare$~\textbf{\color{red}[#1]}}~$\blacksquare$}

\begin{document}

\title{$\zeta$ Ophiuchi as a test bed for modeling accretors}
\author[0000-0002-6718-9472]{M.~Renzo}
\affiliation{Center for Computational Astrophysics, Flatiron Institute, New York, NY 10010, USA}
\affiliation{Department of Physics, Columbia University, New York, NY 10027, USA}
\author{\todo{ }}

\begin{abstract}
\todo{write abstract}
\end{abstract}

\vspace*{-10pt}
\keywords{stars: individual: $\zeta$ Ophiuchi  -- stars: massive --
  stars: binaries} %% check keywords exist


%% main text limit:
%% 3500 words, 50 references, 5 figures (each up to 9 panels)
\section{Introduction}
\label{sec:intro}


\todo{
  In the intro:
  \begin{itemize}
  \item runaway nature
  \item association with pulsar and SNe polluting Earth
  \item debate on parent association
  \item weak wind problem
  \end{itemize}
  Methods:
  \begin{itemize}
  \item self-consistent modeling of the evolution \citep[see also][]{vanrensbergen:96}
  \item depends on many free parameters governing the intricate and
    coupled physics of mass transfer, mixing, rotation
  \end{itemize}
  Aim:
  \begin{itemize}
  \item since observations are not always agreeing with each other, we
    aim at finding a model in the right ballpark and explore how
    physical variations move such model around
  \item in this way we find a set of recommended parameters for the
    evolution of massive binary system going through stable mass transfer
  \end{itemize}
}

The nearest O-type star to Earth is $\zeta$ Ophiuchi, classified as
O9.5{\rm IVnn} type star \citep[e.g.,][]{sota:14} and with a parallax
of 5-8\,milliarcsec \citep[e.g.,][and references
therein]{neuhauser:20}. This star has been the target of many
observations and underpins many open puzzles.


\software{
  \texttt{mesaPlot} \citep{mesaplot},
  \texttt{mesaSDK} \citep{mesasdk},
  \texttt{ipython/jupyter} \citep{ipython},
  \texttt{matplotlib} \citep{matplotlib},
  \texttt{NumPy} \citep{numpy},
  \MESA \citep{paxton:11,paxton:13,paxton:15,paxton:18,paxton:19}
}

\section{Modeling mass transfer with \texttt{MESA}}
\label{sec:methods}

\section{Initial grid of models}

\subsection{Favorite model}

\section{Physical variations}

\section{Discussion}

\section{Conclusions}

\section*{Acknowledgements}



\appendix

\section{\texttt{MESA} setup}
\label{sec:software}
\todo{possibly move to methods}
We use \code{MESA} version 15140 to compute our models.  The
\code{MESA} equation of state (EOS) is a blend of the OPAL \citet{Rogers2002}, SCVH
\citet{Saumon1995}, PTEH \citet{Pols1995}, HELM \citet{Timmes2000},
and PC \citet{Potekhin2010} EOSes. \todo{update EOS}

Radiative opacities are primarily from OPAL \citep{Iglesias1993,
  Iglesias1996}, with low-temperature data from \citet{Ferguson2005}
and the high-temperature, Compton-scattering dominated regime by
\citet{Buchler1976}. Electron conduction opacities are from
\citet{Cassisi2007}.

Nuclear reaction rates are a combination of rates from NACRE
\citep{Angulo1999}, JINA REACLIB \citep{Cyburt2010}, plus additional
tabulated weak reaction rates \citet{Fuller1985, Oda1994,
  Langanke2000}. Screening is included via the prescription of
\citet{Chugunov2007}.  Thermal neutrino loss rates are from
\citet{Itoh1996}. We compute the pre-merger evolution using an
8-isotope $\alpha$-chain nuclear reaction network and switch to a
22-isotope nuclear network for the post-merger evolution.

We evolve our models from the pre-main sequence to the terminal age
main sequence of the most massive $58\,M_\odot$ star, defined as the
time when the central hydrogen abundance $X(^1\mathrm{H})\leq 10^{-4}$.
We treat convection using the Ledoux criterion, and include
thermohaline mixing (until the central temperature
$\log_{10}(T_c/\mathrm{[K]})>9.45$, \citealt{farmer:16}) and semiconvection, both with an
efficiency factor of 1. We assume $\alpha_\mathrm{MLT}=2.0$ and use
\cite{brott:11} overshooting for the convective core burning. We have
tested that varying core overshooting does not impact significantly the
post-merger evolution, however, when including shell overshooting
and/or undershooting we were unable to find solutions to the stellar
structure equations. Moreover, we employ the MLT++ artificial
enhancement of the convective flux \citep[e.g.,][]{paxton:15,
  jiang:15}. Stellar winds are included using the algorithms from
\cite{vink:01} with an efficiency factor of 1.%  and \cite{nugis:00} when the surface en abundance
% is below 0.4 and $T_\mathrm{eff}>10^{4.7}\,\mathrm{K}$. We use a
% hyperbolic tangent interpolation between the two for
% hydrogen-deficient surfaces with
% $10^{4.5}\,\mathrm{K}\leq T_\mathrm{eff}\leq10^{4.7}\,\mathrm{K}$. In
% both cases, we use an efficiency factor of 1, and our post-merger models never
% become cool enough to use a cool wind algorithm \citep[e.g.,][]{renzo:17}.

To compute through the very late phases, we reduce the core resolution
and increase the numerical solver tolerance when the central
temperature increases above $\log_{10}(T_c/\mathrm{[K]})>9.45$. We
define the onset of core-collapse when the iron-core infall velocity
exceeds $1000\,\mathrm{km\ s^{-1}}$ \citep[e.g.,][]{woosley:02}.

The inlists, processing scripts, and model output will be made available at~\href{link}{link}.

\bibliographystyle{aasjournal}
\bibliography{./zeta_oph.bib}

\end{document}
