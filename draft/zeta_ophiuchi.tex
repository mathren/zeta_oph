\documentclass[twocolumn,twocolappendix,trackchanges]{aastex63}
\usepackage{amsmath}
\newcommand{\code}[1]{\texttt{#1}}
\newcommand{\mesa}{\code{MESA}}
\newcommand{\MESA}{\code{MESA}}
\renewcommand{\labelitemii}{$\bullet$}
\newcommand{\kms}{{\mathrm{km\ s^{-1}}}}
\newcommand{\Msun}{{\mathrm{M}_\odot}}
\newcommand{\kev}{\mathrm{keV}}
\newcommand{\gk}{\ensuremath{\,\rm{GK}}}
\usepackage{CJK}
\DeclareRobustCommand{\Eqref}[1]{Eq.~\ref{#1}}
\DeclareRobustCommand{\Figref}[1]{Fig.~\ref{#1}}
\DeclareRobustCommand{\Tabref}[1]{Tab.~\ref{#1}}
\DeclareRobustCommand{\Secref}[1]{Sec.~\ref{#1}}

\newcommand{\todo}[1]{{\large $\blacksquare$~\textbf{\color{red}[#1]}}~$\blacksquare$}

\begin{document}

\title{$\zeta$ Ophiuchi as a test for models of accreting stars in
  massive binaries}
\author[0000-0002-6718-9472]{M.~Renzo}
\affiliation{Department of Physics, Columbia University, New York, NY 10027, USA}
\affiliation{Center for Computational Astrophysics, Flatiron Institute, New York, NY 10010, USA}

\author{\todo{TBD}}

\begin{abstract}
\centering \todo{TBD}
\end{abstract}

\vspace*{-10pt}
\keywords{stars: individual: $\zeta$ Ophiuchi  -- stars: massive --
  stars: binaries} %% check keywords exist


%% main text limit:
%% 3500 words, 50 references, 5 figures (each up to 9 panels)
\section{Introduction}
\label{sec:intro}


The nearest O-type star to Earth is $\zeta$ Ophiuchi \citep[spectral
type O9.5{\rm IVnn},][]{sota:14}, at % with a parallax of
% 5-8\,milliarcsec corresponding to
a distance of $\sim$110\,pc \citep[e.g.,][and references
therein]{neuhauser:20}. It was originally identified as a runaway star
through its large proper motion by \cite{blaauw:52}, who identified
the Scorpio-Centaurus group as its parent association.  Unfortunately,
the \emph{Gaia} data for this object are not of sufficient quality to
improve previous astrometric results, but estimates of the peculiar
velocity range in $30-50\,\kms$
\citep[e.g.,][]{zehe:18, neuhauser:20}. The
large velocity with respect the surrounding interstellar material is
also confirmed by the presence of a prominent bow-shock
\citep[e.g.,][]{bodensteiner:18}.

Because of its young apparent age, extremely fast rotation
($v\sin(i)\sim 400\,\kms$, e.g., \citealt{zehe:18}),
and nitrogen (N) and helium (He) rich surface \citep[e.g.,][]{blaauw:93,
  villamariz:05, marcolino:09}, $\zeta$ Oph is considered a prime
candidate for the binary supernova scenario \citep{blaauw:61,
  renzo:19walk}. In this scenario, after a phase of mass transfer in a
massive binary, the core collapse of the donor star disrupts the
binary, and the former accretor is ejected with its pre-explosion
orbital velocity. Many studies have suggested $\zeta$ Oph
might have accreted mass from a companion before acquiring its large
velocity, both from spectroscopic and kinematic considerations
\citep[e.g.,][]{blaauw:93, hoogerwerf:00, hoogerwerf:01, tetzlaff:10, neuhauser:20}
and using stellar modeling arguments
\citep[e.g.,][]{vanrensbergen:96}.  More specifically,
\cite{neuhauser:20} suggested that the supernova that ejected $\zeta$
Oph produced PSR B1706-16 and also injected the short-lived
radioactive isotope $^{60}\mathrm{Fe}$ on Earth about $\sim 1.5$\,Myr
ago. This argues strongly for a successful supernova explosion
accompanied by a large $\sim 250\,\kms$ natal kick,
which would suffice to disrupt the binary.

Although the nature of $\zeta$ Oph as a binary product is well
established, its large rotation rate has lead most attempts to explain
the surface composition to rely on rotational mixing
\cite[e.g.,][]{maeder:00}. Even the binary models of
\cite{vanrensbergen:96} assumed spin-up due to mass accretion
\citep[e.g.,][]{packet:81} to drive rotational mixing from the
interior of the accreting star (see also
\citealt{cantiello:07}). However, \cite{villamariz:05} were unable to
find good fit for the stellar spectra using the rotating models from
\cite{meynet:00}.

This may not be surprising, since rotational mixing has
lower efficiency for metal-rich and relatively low mass stars because
of the increased importance of mean molecular weight gradients and
longer thermal timescales compared to more massive stars
\citep[e.g.,][]{yoon:06, perna:14}. The parent association of $\zeta$
Oph has $Z=0.01-0.02\simeq Z_\odot$ \citep[e.g.,][]{murphy:21}, and
mass estimates for the star range from $13-25\,M_\odot$, i.e. on the
lower end of the range where efficient mixing might bring He and
CNO-processed material to the surface (chemically homogeneous
evolution).

On top of the surface abundances, its extreme rotation rate, and the
peculiar space velocity, $\zeta$ Oph poses a number of other
puzzles: its wind mass-loss rate is about two orders of magnitude
lower than theoretical predictions (weak wind problem,
\citealt{marcolino:09}), the star exhibits spectral variability with
occasional appearance of H$\alpha$ in emission
\citep[e.g.,][]{walker:79}, and is potentially magnetic \todo{true?ref?}.

\todo{
  Aim:
  \begin{itemize}
    \item importance of binary products and opportunity to understand
  binary physics with $\zeta$ Oph
  \item since observations are not always agreeing with each other, we
    aim at finding a model in the right ballpark and explore how
    physical variations move such model around
  \item in this way we find a set of recommended parameters for the
    evolution of massive binary system going through stable mass transfer
  \end{itemize}
}


Here, we present the first self-consistent binary evolution model for
$\zeta$ Oph computing simultaneously the coupled evolution of donor
and accretor star. After presenting our calculations in
\Secref{sec:methods}, we show our best model which reproduces the
majority of the salient features of this star in
\Secref{sec:best_model}, and discuss the sensitivity of our results to
the admittedly many free parameters required for this kind of
computations in \Secref{sec:param_variations}. Finally, we conclude in
\Secref{sec:conclusions}.






\section{Modeling mass transfer with \texttt{MESA}}
\label{sec:methods}

\todo{  Methods:
  \begin{itemize}
  \item self-consistent modeling of the evolution
  \item depends on many free parameters governing the intricate and
    coupled physics of mass transfer, mixing, rotation
  \end{itemize}
}

\section{Best model}
\label{sec:best_model}

\section{Physical variations}
\label{sec:param_variations}


\section{Conclusions}
\label{sec:conclusions}


\software{
  \texttt{mesaPlot} \citep{mesaplot},
  \texttt{mesaSDK} \citep{mesasdk},
  \texttt{ipython/jupyter} \citep{ipython},
  \texttt{matplotlib} \citep{matplotlib},
  \texttt{NumPy} \citep{numpy},
  \MESA \citep{paxton:11,paxton:13,paxton:15,paxton:18,paxton:19}
}

\section*{Acknowledgements}



\appendix

\section{\texttt{MESA} setup}
\label{sec:software}

\todo{MLT--?}

\todo{possibly move to methods}
We use \code{MESA} version 15140 to compute our models.  The
\code{MESA} equation of state (EOS) is a blend of the OPAL \citet{Rogers2002}, SCVH
\citet{Saumon1995}, PTEH \citet{Pols1995}, HELM \citet{Timmes2000},
and PC \citet{Potekhin2010} EOSes. \todo{check if updated EOS?}

Radiative opacities are primarily from OPAL \citep{Iglesias1993,
  Iglesias1996}, with low-temperature data from \citet{Ferguson2005}
and the high-temperature, Compton-scattering dominated regime by
\citet{Buchler1976}. Electron conduction opacities are from
\citet{Cassisi2007}.

Nuclear reaction rates are a combination of rates from NACRE
\citep{Angulo1999}, JINA REACLIB \citep{Cyburt2010}, plus additional
tabulated weak reaction rates \citet{Fuller1985, Oda1994,
  Langanke2000}. Screening is included via the prescription of
\citet{Chugunov2007}.  Thermal neutrino loss rates are from
\citet{Itoh1996}. We use a
22-isotope nuclear network (\texttt{approx\_21\_plus\_cr56}).

We treat convection using the Ledoux criterion, and include
thermohaline mixing and semiconvection, both with an efficiency factor
of 1. We assume $\alpha_\mathrm{MLT}=1.5$ and use \todo{fix}
\cite{brott:11} overshooting for the convective core
burning. \todo{fix} Moreover, we employ the MLT++ artificial
enhancement of the convective flux \citep[e.g.,][]{paxton:15}. Stellar
winds are included using the algorithms from \cite{vink:01} with an
efficiency factor of
1.% and \cite{nugis:00} when the surface en abundance
% is below 0.4 and $T_\mathrm{eff}>10^{4.7}\,\mathrm{K}$. We use a
% hyperbolic tangent interpolation between the two for
% hydrogen-deficient surfaces with
% $10^{4.5}\,\mathrm{K}\leq T_\mathrm{eff}\leq10^{4.7}\,\mathrm{K}$. In
% both cases, we use an efficiency factor of 1, and our post-merger models never
% become cool enough to use a cool wind algorithm \citep[e.g.,][]{renzo:17}.

The inlists, processing scripts, and model output will be made available at~\href{link}{link}.

\bibliographystyle{aasjournal}
\bibliography{./zeta_ophiuchi.bib}

\end{document}
